%
\documentclass[11pt]{article}
\usepackage[margin=1in]{geometry}                % See geometry.pdf to learn the layout options. There are lots.
\geometry{letterpaper}                   % ... or a4paper or a5paper or ... 
%\geometry{landscape}                % Activate for for rotated page geometry
\usepackage[parfill]{parskip}    % Activate to begin paragraphs with an empty line rather than an indent
\usepackage{graphicx}
\usepackage{amsmath}
\usepackage{amssymb}
\usepackage{epstopdf}
\usepackage{enumerate}
\usepackage{color}
\usepackage{listings}
\usepackage{booktabs}
\usepackage{xfrac}
\usepackage{float}
\usepackage{hyperref}
\lstset{frame=tb,
  language=Java,
  aboveskip=3mm,
  belowskip=3mm,
  showstringspaces=false,
  columns=flexible,
  basicstyle={\small\ttfamily},
  numbers=none,
  numberstyle=\tiny\color{gray},
  keywordstyle=\color{blue},
  commentstyle=\color{dkgreen},
  stringstyle=\color{mauve},
  breaklines=true,
  breakatwhitespace=true
  tabsize=3
}

\DeclareGraphicsRule{.tif}{png}{.png}{`convert #1 `dirname #1`/`basename #1 .tif`.png}

\definecolor{dkgreen}{rgb}{0,0.6,0}
\definecolor{gray}{rgb}{0.5,0.5,0.5}
\definecolor{mauve}{rgb}{0.58,0,0.82}

\newcommand{\noin}{\noindent}    
\newcommand{\logit}{\textrm{logit}} 
\newcommand{\var}{\textrm{Var}}
\newcommand{\cov}{\textrm{Cov}} 
\newcommand{\corr}{\textrm{Corr}} 
\newcommand{\N}{\mathcal{N}}
\newcommand{\NBin}{\textrm{NBin}}
\newcommand{\Bern}{\textrm{Bern}}
\newcommand{\Bin}{\textrm{Bin}}
\newcommand{\Beta}{\textrm{Beta}}
\newcommand{\Gam}{\textrm{Gamma}}
\newcommand{\Expo}{\textrm{Expo}}
\newcommand{\Pois}{\textrm{Pois}}
\newcommand{\Unif}{\textrm{Unif}}
\newcommand{\eq}[1]{\begin{align*}#1\end{align*}}


\title{CS186 Homework 7}
%\subtitle{Counting}
\author{Yuan Jiang and Lewin Xue}
\date{}                                           % Activate to display a given date or no date

\begin{document}
\maketitle

\section*{Problem 1}

\subsection*{Part (a)}


The name ``Mechanical Turk'' comes from a chess-playing automaton created during the 18th century that was called ``The Turk''. Later on, ``The Turk'' was revealed to be not a machine, but a chess master that was hidden in a special compartment controlling its operations. Similarly, AMT allows humans to help today's machines perform tasked that computers are currently unable to do. 

\subsection*{Part (b)}

The last task was a verification of the descriptions of pictures shown to users in hopes that descriptions for items would improve.

\subsection*{Part (c)}


http://www.oregonlive.com/living/index.ssf/2015/02/how_amazons_crowdsourcing_foru.html

The crux of the article talks about how research in the old days collected data from a diverse group of subjects, but now with the development of AMT, more and more researchers are turning to AMT for data collection. While AMT promises a lot of potential for academic research, the downside of AMT is that there are groups of super-users that complete the same HITs over and over again. This brings into question how reliable the data collected from AMT is.

\section*{Problem 2}

\subsection*{Part (a)}
\subsection*{Part (b)}
\subsection*{Part (c)}
\subsection*{Part (d)}

\section*{Problem 3}

\subsection*{Part (a)}
\subsection*{Part (b)}
\subsection*{Part (c)}
\subsection*{Part (d)}

\section*{Problem 4}

\subsection*{Part (a)}
\subsection*{Part (b)}
\subsection*{Part (c)}
\subsection*{Part (d)}

\section*{Problem 6}

\begin{verbatim}
vote_dict = dictionary with keys (image1, image2) (number of votes for image 1 came first before image 2)

function obtain_ranking(vote_dict)

  rankings = {pic1, pic2, ... , pic8}

  //for each pair of compared images we find the number of times picture 1 came before the second
  //We compute the total rank by squaring the percentage of times img 1 came before image 2 and summing over all combinations including img 1.
  //We repeat this for every image
  for (image1, image2) in vote_dict:
    # compute squared ranking - e.g. 0.8 * 8
    rankings[image1] += percentage of votes for image 1/ total number of votes * number of votes 
    rankings[image2} += percentage of votes for image 2/ total number of votes * number of votes 


  return sorted pics by ranking, highest to lowest  
\end{verbatim}


\end{document}